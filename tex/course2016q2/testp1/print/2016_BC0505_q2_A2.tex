
        \documentclass[10pt,brazil,a4paper]{exam}
        \usepackage[latin1]{inputenc}
        \usepackage[portuguese]{babel}
        \usepackage[dvips]{graphicx}
        %\usepackage{multicol}
        %\usepackage{shadow}
        %\usepackage{pifont}
        %\usepackage{listings}
        %\usepackage{fancyvrb}
        
        \newcommand*\varhrulefill[1][0.4pt]{\leavevmode\leaders\hrule height#1\hfill\kern0pt}
        
        \def\drawLines#1{{\color{cyan}\foreach \x in {1,...,#1}{\par\vspace{2mm}\noindent\hrulefill}}}
        
        \usepackage{enumitem}
        \usepackage{multirow}
        \usepackage{amsmath}
        \usepackage{changepage,ifthen}
        %\usepackage{boxedminipage}
        %\usepackage{theorem}
        \usepackage{verbatim}
        \usepackage{tabularx}
        %\usepackage{moreverb}
        % Font selection
        \usepackage[T1]{fontenc}
        \usepackage{times} 
        
        \usepackage{multido}  % border

        %\usepackage{relsize}
        \usepackage{pst-barcode}
        \usepackage{tikz}
        \setlength{\textwidth}{185mm}
        \setlength{\oddsidemargin}{-0.5in}
        \setlength{\evensidemargin}{0in}
        \setlength{\columnsep}{8mm}
        \setlength{\topmargin}{-28mm}
        \setlength{\textheight}{272mm}
        \setlength{\itemsep}{0in}
        \begin{document}
        \pagestyle{empty}
        %\lstset{language=python}
        \makeatletter\renewcommand*\cleardoublepage{\ifodd\c@page \else\hbox{}\newpage\fi}
\makeatother
\cleardoublepage
\begin{table}[h]\centering
\begin{tabular}{|p{16mm}|p{16cm}|}
\hline\multirow{4}{*}{\hspace{-2mm}\includegraphics[width=2cm]{../../../../figs/ufabc.eps}}
&\vspace{-2mm}\noindent\large\textbf{Universidade Federal do ABC}\\
&\noindent\textbf{Course:} Processamento da Informa��o - BC0505\\
&\noindent\textbf{Teacher(s):} Francisco de Assis Zampirolli\\
&\noindent\textbf{Period:} 2/2016\hfill\textbf{Modality:} Presencial\hfill\textbf{Date:} 22/06/2016\hspace{-8mm}\\
\hline\end{tabular}
\end{table}

\vspace{-5mm}
\leavevmode\put(-24,0){\color{black}\circle*{15}}\hspace{-5mm}  \leavevmode\put(535,0){\color{black}\circle*{15}}\vspace{2mm}

\Large{
\noindent\textbf{Student:}  Fulano Junior
\hfill\textbf{Registration:} 11000123
\hfill\textbf{Room:} 2016$\_$BC0505$\_$q2$\_$A2
\hspace{-1mm}

\vspace{8mm}
\noindent\textbf{Sig.:}\rule{11.5cm}{0.1pt}\hfill\hspace{1cm}
}\normalsize\begin{pspicture}(6,0in)
\psframe[linecolor=black,fillstyle=solid,fillcolor=white](-0.4,-0.2)(5.7,1.1)\psbarcode[scalex=1.6,scaley=0.35]{000011000123}{}{ean13}
\end{pspicture}
\vspace{-2mm}\begin{center}\textbf{Parte 1 - Answer Sheet - Do not use this sheet as a draft!}\end{center}
\begin{center}
\begin{tikzpicture}[font=\tiny]
  \foreach \letter/\position in {A/1,B/2,C/3,D/4,E/5} {
    \node[inner sep=3pt] at ({\position * 0.5},0) {\letter};
  }
  \foreach \line in {1,...,9} {
     \begin{scope}[xshift=0cm,yshift=-(\line-1+1)*5mm]
       \foreach \letter/\position in {A/1,B/2,C/3,D/4,E/5} {
           \node at (-0.1,0) {\line};
           \node[fill=black!100,draw,circle,inner sep=3pt] at ({\position * 0.5},0) {};
           \node[fill=white,draw,circle,inner sep=2pt] at ({\position * 0.5},0) {};
       }
     \end{scope}
  }
\end{tikzpicture}\hspace{1 cm}

\end{center}


\vspace{-5mm}
\leavevmode\put(-24,0){\color{black}\circle*{15}}\hspace{-5mm}\hrulefill  \leavevmode\put(10,0){\color{black}\circle*{15}}
\begin{table}[h]
\centering
\textbf{Calculation of concepts} \\ \vspace{5mm}
\begin{tabular}{|c|c|c|c|} \hline
Questions 1-12 & Textual Questions & Final\\ \hline
& &  \\
& &  \\
& &  \\ \hline
\end{tabular}
\end{table}

 \vspace{5mm} {\small \noindent \textbf{Instructions: }
\vspace{-1mm}\begin{enumerate}[label=\alph*)]
\itemsep0pt\parskip0pt\parsep0pt

\item Paint only INSIDE OF THE CIRCLES of each question.
\item No Rasure.
\item Each question has only one correct answer.
\item It will be considered only the answers in "Part 1" area on this page to the questions of multiple choice.
\end{enumerate}
\vspace{2mm}{\normalsize\noindent\textbf{Parte 2 - Multiple-Choice Questions}}
{\normalsize
\begin{questions}
\itemsep0pt\parskip0pt\parsep0pt
\question pergunta Q2  com exemplo de f�rmula em tex: 
 $\sin A \cos B = \frac{1}{2}\left[ \sin(A-B)+\sin(A+B) \right]$
\begin{oneparchoices}
\itemsep0pt\parskip0pt\parsep0pt
\choice resposta 2b\choice resposta 2d\choice resposta 2e\choice resposta 2c\choice resposta 2a\end{oneparchoices}\vspace{0mm}
\question pergunta f�cil Q1b1 - com primeira varia��es da subclasse b:
\begin{oneparchoices}
\itemsep0pt\parskip0pt\parsep0pt
\choice resposta 1b1-b\choice resposta 1b1-c\choice resposta 1b1-e\choice resposta 1b1-a\choice resposta 1b1-d\end{oneparchoices}\vspace{0mm}
\question pergunta Q14
\begin{oneparchoices}
\itemsep0pt\parskip0pt\parsep0pt
\choice resposta 14b\choice resposta 14a\choice resposta 14c\choice resposta 14e\choice resposta 14d\end{oneparchoices}\vspace{0mm}
\question pergunta Q3
\begin{oneparchoices}
\itemsep0pt\parskip0pt\parsep0pt
\choice resposta 3d\choice resposta 3c\choice resposta 3a\choice resposta 3e\choice resposta 3b\end{oneparchoices}\vspace{0mm}
\question pergunta Q6
\begin{oneparchoices}
\itemsep0pt\parskip0pt\parsep0pt
\choice resposta 6a\choice resposta 6e\choice resposta 6d\choice resposta 6c\choice resposta 6b\end{oneparchoices}\vspace{0mm}
\question pergunta Q4
\begin{oneparchoices}
\itemsep0pt\parskip0pt\parsep0pt
\choice resposta 4e\choice resposta 4b\choice resposta 4d\choice resposta 4c\choice resposta 4a\end{oneparchoices}\vspace{0mm}
\question pergunta Q5
\begin{oneparchoices}
\itemsep0pt\parskip0pt\parsep0pt
\choice resposta 5c\choice resposta 5d\choice resposta 5e\choice resposta 5b\choice resposta 5a\end{oneparchoices}\vspace{0mm}
\question pergunta Q11
\begin{oneparchoices}
\itemsep0pt\parskip0pt\parsep0pt
\choice resposta 11a\choice resposta 11d\choice resposta 11b\choice resposta 11e\choice resposta 11c\end{oneparchoices}\vspace{0mm}
\question pergunta Q10
\begin{oneparchoices}
\itemsep0pt\parskip0pt\parsep0pt
\choice resposta 0b\choice resposta 0d\choice resposta 0e\choice resposta 0a\choice resposta 0c\end{oneparchoices}\vspace{0mm}
\end{questions}
}\vspace{2mm}{\normalsize\noindent\textbf{Parte 3 - Textual Questions
}}\\
{\normalsize
\makeatletter\renewcommand*\cleardoublepage{\ifodd\c@page \else\hbox{}\newpage\fi}
\makeatother
\cleardoublepage
\begin{table}[h]\centering
\begin{tabular}{|p{16mm}|p{16cm}|}
\hline\multirow{4}{*}{\hspace{-2mm}\includegraphics[width=2cm]{../../../../figs/ufabc.eps}}
&\vspace{-2mm}\noindent\large\textbf{Universidade Federal do ABC}\\
&\noindent\textbf{Course:} Processamento da Informa��o - BC0505\\
&\noindent\textbf{Teacher(s):} Francisco de Assis Zampirolli\\
&\noindent\textbf{Period:} 2/2016\hfill\textbf{Modality:} Presencial\hfill\textbf{Date:} 22/06/2016\hspace{-8mm}\\
\hline\end{tabular}
\end{table}

\vspace{-5mm}
\leavevmode\put(-24,0){\color{black}\circle*{15}}\hspace{-5mm}  \leavevmode\put(535,0){\color{black}\circle*{15}}\vspace{2mm}

\Large{
\noindent\textbf{Student:}  Fulano Junior
\hfill\textbf{Registration:} 11000123
\hfill\textbf{Room:} 2016$\_$BC0505$\_$q2$\_$A2
\hspace{-1mm}

\vspace{8mm}
\noindent\textbf{Sig.:}\rule{11.5cm}{0.1pt}\hfill\hspace{1cm}
}\normalsize

\vspace{4mm}\noindent Escreva 222 um programa para inverter os elementos �mpares que est�o nas posi��es pares de um vetor com X elementos, onde X � um inteiro definido pelo usu�rio.
 
 \drawLines{10} 


 \ \ \ 
 \newpage
\makeatletter\renewcommand*\cleardoublepage{\ifodd\c@page \else\hbox{}\newpage\fi}
\makeatother
\cleardoublepage
\begin{table}[h]\centering
\begin{tabular}{|p{16mm}|p{16cm}|}
\hline\multirow{4}{*}{\hspace{-2mm}\includegraphics[width=2cm]{../../../../figs/ufabc.eps}}
&\vspace{-2mm}\noindent\large\textbf{Universidade Federal do ABC}\\
&\noindent\textbf{Course:} Processamento da Informa��o - BC0505\\
&\noindent\textbf{Teacher(s):} Francisco de Assis Zampirolli\\
&\noindent\textbf{Period:} 2/2016\hfill\textbf{Modality:} Presencial\hfill\textbf{Date:} 22/06/2016\hspace{-8mm}\\
\hline\end{tabular}
\end{table}

\vspace{-5mm}
\leavevmode\put(-24,0){\color{black}\circle*{15}}\hspace{-5mm}  \leavevmode\put(535,0){\color{black}\circle*{15}}\vspace{2mm}

\Large{
\noindent\textbf{Student:}  Fulano Junior
\hfill\textbf{Registration:} 11000123
\hfill\textbf{Room:} 2016$\_$BC0505$\_$q2$\_$A2
\hspace{-1mm}

\vspace{8mm}
\noindent\textbf{Sig.:}\rule{11.5cm}{0.1pt}\hfill\hspace{1cm}
}\normalsize

\vspace{4mm}\noindent Escreva 444 um programa para inverter os elementos �mpares que est�o nas posi��es pares de um vetor com X elementos, onde X � um inteiro definido pelo usu�rio.
 
 \drawLines{10} 


 \ \ \ 
 \newpage
}
\makeatletter\renewcommand*\cleardoublepage{\ifodd\c@page \else\hbox{}\newpage\fi}
\makeatother
\cleardoublepage
\begin{table}[h]\centering
\begin{tabular}{|p{16mm}|p{16cm}|}
\hline\multirow{4}{*}{\hspace{-2mm}\includegraphics[width=2cm]{../../../../figs/ufabc.eps}}
&\vspace{-2mm}\noindent\large\textbf{Universidade Federal do ABC}\\
&\noindent\textbf{Course:} Processamento da Informa��o - BC0505\\
&\noindent\textbf{Teacher(s):} Francisco de Assis Zampirolli\\
&\noindent\textbf{Period:} 2/2016\hfill\textbf{Modality:} Presencial\hfill\textbf{Date:} 22/06/2016\hspace{-8mm}\\
\hline\end{tabular}
\end{table}

\vspace{-5mm}
\leavevmode\put(-24,0){\color{black}\circle*{15}}\hspace{-5mm}  \leavevmode\put(535,0){\color{black}\circle*{15}}\vspace{2mm}

\Large{
\noindent\textbf{Student:}  Gustavo Neto
\hfill\textbf{Registration:} 11000111
\hfill\textbf{Room:} 2016$\_$BC0505$\_$q2$\_$A2
\hspace{-1mm}

\vspace{8mm}
\noindent\textbf{Sig.:}\rule{11.5cm}{0.1pt}\hfill\hspace{1cm}
}\normalsize\begin{pspicture}(6,0in)
\psframe[linecolor=black,fillstyle=solid,fillcolor=white](-0.4,-0.2)(5.7,1.1)\psbarcode[scalex=1.6,scaley=0.35]{000011000111}{}{ean13}
\end{pspicture}
\vspace{-2mm}\begin{center}\textbf{Parte 1 - Answer Sheet - Do not use this sheet as a draft!}\end{center}
\begin{center}
\begin{tikzpicture}[font=\tiny]
  \foreach \letter/\position in {A/1,B/2,C/3,D/4,E/5} {
    \node[inner sep=3pt] at ({\position * 0.5},0) {\letter};
  }
  \foreach \line in {1,...,9} {
     \begin{scope}[xshift=0cm,yshift=-(\line-1+1)*5mm]
       \foreach \letter/\position in {A/1,B/2,C/3,D/4,E/5} {
           \node at (-0.1,0) {\line};
           \node[fill=black!100,draw,circle,inner sep=3pt] at ({\position * 0.5},0) {};
           \node[fill=white,draw,circle,inner sep=2pt] at ({\position * 0.5},0) {};
       }
     \end{scope}
  }
\end{tikzpicture}\hspace{1 cm}

\end{center}


\vspace{-5mm}
\leavevmode\put(-24,0){\color{black}\circle*{15}}\hspace{-5mm}\hrulefill  \leavevmode\put(10,0){\color{black}\circle*{15}}
\begin{table}[h]
\centering
\textbf{Calculation of concepts} \\ \vspace{5mm}
\begin{tabular}{|c|c|c|c|} \hline
Questions 1-12 & Textual Questions & Final\\ \hline
& &  \\
& &  \\
& &  \\ \hline
\end{tabular}
\end{table}

 \vspace{5mm} {\small \noindent \textbf{Instructions: }
\vspace{-1mm}\begin{enumerate}[label=\alph*)]
\itemsep0pt\parskip0pt\parsep0pt

\item Paint only INSIDE OF THE CIRCLES of each question.
\item No Rasure.
\item Each question has only one correct answer.
\item It will be considered only the answers in "Part 1" area on this page to the questions of multiple choice.
\end{enumerate}
\vspace{2mm}{\normalsize\noindent\textbf{Parte 2 - Multiple-Choice Questions}}
{\normalsize
\begin{questions}
\itemsep0pt\parskip0pt\parsep0pt
\question pergunta Q13
\begin{oneparchoices}
\itemsep0pt\parskip0pt\parsep0pt
\choice resposta 13c\choice resposta 13d\choice resposta 13e\choice resposta 13a\choice resposta 13b\end{oneparchoices}\vspace{0mm}
\question pergunta Q2  com exemplo de f�rmula em tex: 
 $\sin A \cos B = \frac{1}{2}\left[ \sin(A-B)+\sin(A+B) \right]$
\begin{oneparchoices}
\itemsep0pt\parskip0pt\parsep0pt
\choice resposta 2c\choice resposta 2d\choice resposta 2a\choice resposta 2e\choice resposta 2b\end{oneparchoices}\vspace{0mm}
\question pergunta Q3
\begin{oneparchoices}
\itemsep0pt\parskip0pt\parsep0pt
\choice resposta 3b\choice resposta 3d\choice resposta 3e\choice resposta 3c\choice resposta 3a\end{oneparchoices}\vspace{0mm}
\question pergunta Q14
\begin{oneparchoices}
\itemsep0pt\parskip0pt\parsep0pt
\choice resposta 14e\choice resposta 14c\choice resposta 14a\choice resposta 14b\choice resposta 14d\end{oneparchoices}\vspace{0mm}
\question pergunta Q12
\begin{oneparchoices}
\itemsep0pt\parskip0pt\parsep0pt
\choice resposta 12a\choice resposta 12d\choice resposta 12e\choice resposta 12c\choice resposta 12b\end{oneparchoices}\vspace{0mm}
\question pergunta Q5
\begin{oneparchoices}
\itemsep0pt\parskip0pt\parsep0pt
\choice resposta 5c\choice resposta 5e\choice resposta 5a\choice resposta 5b\choice resposta 5d\end{oneparchoices}\vspace{0mm}
\question pergunta Q6
\begin{oneparchoices}
\itemsep0pt\parskip0pt\parsep0pt
\choice resposta 6b\choice resposta 6c\choice resposta 6d\choice resposta 6e\choice resposta 6a\end{oneparchoices}\vspace{0mm}
\question pergunta Q7
\begin{oneparchoices}
\itemsep0pt\parskip0pt\parsep0pt
\choice resposta 7b\choice resposta 7e\choice resposta 7c\choice resposta 7a\choice resposta 7d\end{oneparchoices}\vspace{0mm}
\question pergunta Q11
\begin{oneparchoices}
\itemsep0pt\parskip0pt\parsep0pt
\choice resposta 11b\choice resposta 11d\choice resposta 11a\choice resposta 11e\choice resposta 11c\end{oneparchoices}\vspace{0mm}
\end{questions}
}\vspace{2mm}{\normalsize\noindent\textbf{Parte 3 - Textual Questions
}}\\
{\normalsize
\makeatletter\renewcommand*\cleardoublepage{\ifodd\c@page \else\hbox{}\newpage\fi}
\makeatother
\cleardoublepage
\begin{table}[h]\centering
\begin{tabular}{|p{16mm}|p{16cm}|}
\hline\multirow{4}{*}{\hspace{-2mm}\includegraphics[width=2cm]{../../../../figs/ufabc.eps}}
&\vspace{-2mm}\noindent\large\textbf{Universidade Federal do ABC}\\
&\noindent\textbf{Course:} Processamento da Informa��o - BC0505\\
&\noindent\textbf{Teacher(s):} Francisco de Assis Zampirolli\\
&\noindent\textbf{Period:} 2/2016\hfill\textbf{Modality:} Presencial\hfill\textbf{Date:} 22/06/2016\hspace{-8mm}\\
\hline\end{tabular}
\end{table}

\vspace{-5mm}
\leavevmode\put(-24,0){\color{black}\circle*{15}}\hspace{-5mm}  \leavevmode\put(535,0){\color{black}\circle*{15}}\vspace{2mm}

\Large{
\noindent\textbf{Student:}  Gustavo Neto
\hfill\textbf{Registration:} 11000111
\hfill\textbf{Room:} 2016$\_$BC0505$\_$q2$\_$A2
\hspace{-1mm}

\vspace{8mm}
\noindent\textbf{Sig.:}\rule{11.5cm}{0.1pt}\hfill\hspace{1cm}
}\normalsize

\vspace{4mm}\noindent Escreva 222 um programa para inverter os elementos �mpares que est�o nas posi��es pares de um vetor com X elementos, onde X � um inteiro definido pelo usu�rio.
 
 \drawLines{10} 


 \ \ \ 
 \newpage
\makeatletter\renewcommand*\cleardoublepage{\ifodd\c@page \else\hbox{}\newpage\fi}
\makeatother
\cleardoublepage
\begin{table}[h]\centering
\begin{tabular}{|p{16mm}|p{16cm}|}
\hline\multirow{4}{*}{\hspace{-2mm}\includegraphics[width=2cm]{../../../../figs/ufabc.eps}}
&\vspace{-2mm}\noindent\large\textbf{Universidade Federal do ABC}\\
&\noindent\textbf{Course:} Processamento da Informa��o - BC0505\\
&\noindent\textbf{Teacher(s):} Francisco de Assis Zampirolli\\
&\noindent\textbf{Period:} 2/2016\hfill\textbf{Modality:} Presencial\hfill\textbf{Date:} 22/06/2016\hspace{-8mm}\\
\hline\end{tabular}
\end{table}

\vspace{-5mm}
\leavevmode\put(-24,0){\color{black}\circle*{15}}\hspace{-5mm}  \leavevmode\put(535,0){\color{black}\circle*{15}}\vspace{2mm}

\Large{
\noindent\textbf{Student:}  Gustavo Neto
\hfill\textbf{Registration:} 11000111
\hfill\textbf{Room:} 2016$\_$BC0505$\_$q2$\_$A2
\hspace{-1mm}

\vspace{8mm}
\noindent\textbf{Sig.:}\rule{11.5cm}{0.1pt}\hfill\hspace{1cm}
}\normalsize

\vspace{4mm}\noindent O programa abaixo l� dois valores para as vari�veis X e Y, efetua a troca dos valores de forma que a vari�vel X passe a ter o valor de Y, e que a vari�vel Y passe a ter o valor de X. Complete a(s) instru��o(�es) "AQUI".
 \begin{verbatim}
 programa
 {
     funcao inicio()
     {
           real X, Y, aux
           leia (X, Y)
           AQUI
           escreva(X, Y)
      }
 }
 \end{verbatim}
 
 \drawLines{10} 


 \ \ \ 
 \newpage
}
\end{document}