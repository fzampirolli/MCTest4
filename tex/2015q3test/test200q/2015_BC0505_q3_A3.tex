
        \documentclass[10pt,brazil,a4paper]{exam}
        \usepackage[latin1]{inputenc}
        \usepackage[portuguese]{babel}
        \usepackage[dvips]{graphicx}
        %\usepackage{multicol}
        %\usepackage{shadow}
        %\usepackage{pifont}
        %\usepackage{listings}
        %\usepackage{fancyvrb}
        
        \newcommand*\varhrulefill[1][0.4pt]{\leavevmode\leaders\hrule height#1\hfill\kern0pt}
        
        \def\drawLines#1{{\color{cyan}\foreach \x in {1,...,#1}{\par\vspace{2mm}\noindent\hrulefill}}}
        
        \usepackage{enumitem}
        \usepackage{multirow}
        \usepackage{amsmath}
        \usepackage{changepage,ifthen}
        %\usepackage{boxedminipage}
        %\usepackage{theorem}
        \usepackage{verbatim}
        \usepackage{tabularx}
        %\usepackage{moreverb}
        \usepackage{times}
        %\usepackage{relsize}
        \usepackage{pst-barcode}
        \usepackage{tikz}
        \setlength{\textwidth}{185mm}
        \setlength{\oddsidemargin}{-0.5in}
        \setlength{\evensidemargin}{0in}
        \setlength{\columnsep}{8mm}
        \setlength{\topmargin}{-28mm}
        \setlength{\textheight}{265mm}
        \setlength{\itemsep}{0in}
        \begin{document}
        \pagestyle{empty}
        %\lstset{language=python}
        \begin{table}[h]\centering
\begin{tabular}{|p{16mm}|p{16cm}|}
\hline\multirow{4}{*}{\hspace{-2mm}\includegraphics[width=2cm]{../../../figs/ufabc.eps}}
&\vspace{-2mm}\noindent\large\textbf{Universidade Federal do ABC}\\
&\noindent\textbf{Disciplina:} Processamento da Informa��o - BC0505\\
&\noindent\textbf{Professor(es):} Denise Goya, Itana Stiubiener, Francisco Zampirolli, Luiz Rozante, Monael Ribeiro\\
&\noindent\textbf{Quadrimestre:} 3/2015\hfill\textbf{Modalidade:} Semipresencial\hfill\textbf{Data:} 22/10/2015\hspace{-8mm}\\
\hline\end{tabular}
\end{table}
\vspace{-4mm}\small{
\noindent\textbf{Aluno:} Joao Manoel
\hfill\textbf{Matr�cula:} 11111111
\hfill\textbf{Turma:} 2015$\_$BC0505$\_$q3$\_$A3
\hspace{-1mm}

\vspace{8mm}
\noindent\textbf{Ass:}\rule{11.5cm}{0.1pt}\hfill\hspace{1cm}
}\begin{pspicture}(6,0in)
\psbarcode[scalex=1.6,scaley=0.35]{000011111111}{}{ean13}
\end{pspicture}
\\{\scriptsize

\noindent\textbf{Instru��es: }\vspace{-1mm}\begin{verbatim}

1. Pinte somente DENTRO DOS C�RCULOS de cada quest�o.  2. N�o rasure.  3. Cada quest�o possui uma �nica resposta correta.
4. Somente ser�o consideradas as respostas na regi�o "Parte 1" desta p�gina para as quest�es de m�ltipa-escolha.
\end{verbatim}
\begin{center}\textbf{Parte 1 - Quadro(s) de Respostas - N�o utilize esta FOLHA como rascunho!}\end{center}
\vspace{-5mm}\noindent\varhrulefill[0.4mm]\vspace{1mm}

\vspace{-3mm}\noindent\varhrulefill[0.4mm]\vspace{1mm}

\begin{center}
\begin{tikzpicture}[font=\tiny]
  \foreach \letter/\position in {A/1,B/2,C/3,D/4,E/5} {
    \node[inner sep=3pt] at ({\position * 0.5},0) {\letter};
  }
  \foreach \line in {1,...,35} {
     \begin{scope}[xshift=0cm,yshift=-(\line-1+1)*5mm]
       \foreach \letter/\position in {A/1,B/2,C/3,D/4,E/5} {
           \node at (-0.1,0) {\line};
           \node[fill=black!30,draw,circle,inner sep=3pt] at ({\position * 0.5},0) {};
           \node[fill=white,draw,circle,inner sep=2pt] at ({\position * 0.5},0) {};
       }
     \end{scope}
  }
\end{tikzpicture}\hspace{1 cm}
\begin{tikzpicture}[font=\tiny]
  \foreach \letter/\position in {A/1,B/2,C/3,D/4,E/5} {
    \node[inner sep=3pt] at ({\position * 0.5},0) {\letter};
  }
  \foreach \line in {36,...,70} {
     \begin{scope}[xshift=0cm,yshift=-(\line-36+1)*5mm]
       \foreach \letter/\position in {A/1,B/2,C/3,D/4,E/5} {
           \node at (-0.1,0) {\line};
           \node[fill=black!30,draw,circle,inner sep=3pt] at ({\position * 0.5},0) {};
           \node[fill=white,draw,circle,inner sep=2pt] at ({\position * 0.5},0) {};
       }
     \end{scope}
  }
\end{tikzpicture}\hspace{1 cm}
\begin{tikzpicture}[font=\tiny]
  \foreach \letter/\position in {A/1,B/2,C/3,D/4,E/5} {
    \node[inner sep=3pt] at ({\position * 0.5},0) {\letter};
  }
  \foreach \line in {71,...,105} {
     \begin{scope}[xshift=0cm,yshift=-(\line-71+1)*5mm]
       \foreach \letter/\position in {A/1,B/2,C/3,D/4,E/5} {
           \node at (-0.1,0) {\line};
           \node[fill=black!30,draw,circle,inner sep=3pt] at ({\position * 0.5},0) {};
           \node[fill=white,draw,circle,inner sep=2pt] at ({\position * 0.5},0) {};
       }
     \end{scope}
  }
\end{tikzpicture}\hspace{1 cm}
\begin{tikzpicture}[font=\tiny]
  \foreach \letter/\position in {A/1,B/2,C/3,D/4,E/5} {
    \node[inner sep=3pt] at ({\position * 0.5},0) {\letter};
  }
  \foreach \line in {106,...,140} {
     \begin{scope}[xshift=0cm,yshift=-(\line-106+1)*5mm]
       \foreach \letter/\position in {A/1,B/2,C/3,D/4,E/5} {
           \node at (-0.1,0) {\line};
           \node[fill=black!30,draw,circle,inner sep=3pt] at ({\position * 0.5},0) {};
           \node[fill=white,draw,circle,inner sep=2pt] at ({\position * 0.5},0) {};
       }
     \end{scope}
  }
\end{tikzpicture}\hspace{1 cm}


\end{center}
\vspace{1cm}\noindent\varhrulefill[0.4mm]\vspace{1mm}

\vspace{-3mm}\noindent\varhrulefill[0.4mm]\vspace{1mm}


\newpage\begin{table}[h]\centering
\begin{tabular}{|p{16mm}|p{16cm}|}
\hline\multirow{4}{*}{\hspace{-2mm}\includegraphics[width=2cm]{../../../figs/ufabc.eps}}
&\vspace{-2mm}\noindent\large\textbf{Universidade Federal do ABC}\\
&\noindent\textbf{Disciplina:} Processamento da Informa��o - BC0505\\
&\noindent\textbf{Professor(es):} Denise Goya, Itana Stiubiener, Francisco Zampirolli, Luiz Rozante, Monael Ribeiro\\
&\noindent\textbf{Quadrimestre:} 3/2015\hfill\textbf{Modalidade:} Semipresencial\hfill\textbf{Data:} 22/10/2015\hspace{-8mm}\\
\hline\end{tabular}
\end{table}
\vspace{-4mm}\small{
\noindent\textbf{Aluno:} Manoel Sobrenome
\hfill\textbf{Matr�cula:} 11000000
\hfill\textbf{Turma:} 2015$\_$BC0505$\_$q3$\_$A3
\hspace{-1mm}

\vspace{8mm}
\noindent\textbf{Ass:}\rule{11.5cm}{0.1pt}\hfill\hspace{1cm}
}\begin{pspicture}(6,0in)
\psbarcode[scalex=1.6,scaley=0.35]{000011000000}{}{ean13}
\end{pspicture}
\\{\scriptsize

\noindent\textbf{Instru��es: }\vspace{-1mm}\begin{verbatim}

1. Pinte somente DENTRO DOS C�RCULOS de cada quest�o.  2. N�o rasure.  3. Cada quest�o possui uma �nica resposta correta.
4. Somente ser�o consideradas as respostas na regi�o "Parte 1" desta p�gina para as quest�es de m�ltipa-escolha.
\end{verbatim}
\begin{center}\textbf{Parte 1 - Quadro(s) de Respostas - N�o utilize esta FOLHA como rascunho!}\end{center}
\vspace{-5mm}\noindent\varhrulefill[0.4mm]\vspace{1mm}

\vspace{-3mm}\noindent\varhrulefill[0.4mm]\vspace{1mm}

\begin{center}
\begin{tikzpicture}[font=\tiny]
  \foreach \letter/\position in {A/1,B/2,C/3,D/4,E/5} {
    \node[inner sep=3pt] at ({\position * 0.5},0) {\letter};
  }
  \foreach \line in {1,...,35} {
     \begin{scope}[xshift=0cm,yshift=-(\line-1+1)*5mm]
       \foreach \letter/\position in {A/1,B/2,C/3,D/4,E/5} {
           \node at (-0.1,0) {\line};
           \node[fill=black!30,draw,circle,inner sep=3pt] at ({\position * 0.5},0) {};
           \node[fill=white,draw,circle,inner sep=2pt] at ({\position * 0.5},0) {};
       }
     \end{scope}
  }
\end{tikzpicture}\hspace{1 cm}
\begin{tikzpicture}[font=\tiny]
  \foreach \letter/\position in {A/1,B/2,C/3,D/4,E/5} {
    \node[inner sep=3pt] at ({\position * 0.5},0) {\letter};
  }
  \foreach \line in {36,...,70} {
     \begin{scope}[xshift=0cm,yshift=-(\line-36+1)*5mm]
       \foreach \letter/\position in {A/1,B/2,C/3,D/4,E/5} {
           \node at (-0.1,0) {\line};
           \node[fill=black!30,draw,circle,inner sep=3pt] at ({\position * 0.5},0) {};
           \node[fill=white,draw,circle,inner sep=2pt] at ({\position * 0.5},0) {};
       }
     \end{scope}
  }
\end{tikzpicture}\hspace{1 cm}
\begin{tikzpicture}[font=\tiny]
  \foreach \letter/\position in {A/1,B/2,C/3,D/4,E/5} {
    \node[inner sep=3pt] at ({\position * 0.5},0) {\letter};
  }
  \foreach \line in {71,...,105} {
     \begin{scope}[xshift=0cm,yshift=-(\line-71+1)*5mm]
       \foreach \letter/\position in {A/1,B/2,C/3,D/4,E/5} {
           \node at (-0.1,0) {\line};
           \node[fill=black!30,draw,circle,inner sep=3pt] at ({\position * 0.5},0) {};
           \node[fill=white,draw,circle,inner sep=2pt] at ({\position * 0.5},0) {};
       }
     \end{scope}
  }
\end{tikzpicture}\hspace{1 cm}
\begin{tikzpicture}[font=\tiny]
  \foreach \letter/\position in {A/1,B/2,C/3,D/4,E/5} {
    \node[inner sep=3pt] at ({\position * 0.5},0) {\letter};
  }
  \foreach \line in {106,...,140} {
     \begin{scope}[xshift=0cm,yshift=-(\line-106+1)*5mm]
       \foreach \letter/\position in {A/1,B/2,C/3,D/4,E/5} {
           \node at (-0.1,0) {\line};
           \node[fill=black!30,draw,circle,inner sep=3pt] at ({\position * 0.5},0) {};
           \node[fill=white,draw,circle,inner sep=2pt] at ({\position * 0.5},0) {};
       }
     \end{scope}
  }
\end{tikzpicture}\hspace{1 cm}


\end{center}
\vspace{1cm}\noindent\varhrulefill[0.4mm]\vspace{1mm}

\vspace{-3mm}\noindent\varhrulefill[0.4mm]\vspace{1mm}


\newpage\begin{table}[h]\centering
\begin{tabular}{|p{16mm}|p{16cm}|}
\hline\multirow{4}{*}{\hspace{-2mm}\includegraphics[width=2cm]{../../../figs/ufabc.eps}}
&\vspace{-2mm}\noindent\large\textbf{Universidade Federal do ABC}\\
&\noindent\textbf{Disciplina:} Processamento da Informa��o - BC0505\\
&\noindent\textbf{Professor(es):} Denise Goya, Itana Stiubiener, Francisco Zampirolli, Luiz Rozante, Monael Ribeiro\\
&\noindent\textbf{Quadrimestre:} 3/2015\hfill\textbf{Modalidade:} Semipresencial\hfill\textbf{Data:} 22/10/2015\hspace{-8mm}\\
\hline\end{tabular}
\end{table}
\vspace{-4mm}\small{
\noindent\textbf{Aluno:} Maria Bela
\hfill\textbf{Matr�cula:} 11000001
\hfill\textbf{Turma:} 2015$\_$BC0505$\_$q3$\_$A3
\hspace{-1mm}

\vspace{8mm}
\noindent\textbf{Ass:}\rule{11.5cm}{0.1pt}\hfill\hspace{1cm}
}\begin{pspicture}(6,0in)
\psbarcode[scalex=1.6,scaley=0.35]{000011000001}{}{ean13}
\end{pspicture}
\\{\scriptsize

\noindent\textbf{Instru��es: }\vspace{-1mm}\begin{verbatim}

1. Pinte somente DENTRO DOS C�RCULOS de cada quest�o.  2. N�o rasure.  3. Cada quest�o possui uma �nica resposta correta.
4. Somente ser�o consideradas as respostas na regi�o "Parte 1" desta p�gina para as quest�es de m�ltipa-escolha.
\end{verbatim}
\begin{center}\textbf{Parte 1 - Quadro(s) de Respostas - N�o utilize esta FOLHA como rascunho!}\end{center}
\vspace{-5mm}\noindent\varhrulefill[0.4mm]\vspace{1mm}

\vspace{-3mm}\noindent\varhrulefill[0.4mm]\vspace{1mm}

\begin{center}
\begin{tikzpicture}[font=\tiny]
  \foreach \letter/\position in {A/1,B/2,C/3,D/4,E/5} {
    \node[inner sep=3pt] at ({\position * 0.5},0) {\letter};
  }
  \foreach \line in {1,...,35} {
     \begin{scope}[xshift=0cm,yshift=-(\line-1+1)*5mm]
       \foreach \letter/\position in {A/1,B/2,C/3,D/4,E/5} {
           \node at (-0.1,0) {\line};
           \node[fill=black!30,draw,circle,inner sep=3pt] at ({\position * 0.5},0) {};
           \node[fill=white,draw,circle,inner sep=2pt] at ({\position * 0.5},0) {};
       }
     \end{scope}
  }
\end{tikzpicture}\hspace{1 cm}
\begin{tikzpicture}[font=\tiny]
  \foreach \letter/\position in {A/1,B/2,C/3,D/4,E/5} {
    \node[inner sep=3pt] at ({\position * 0.5},0) {\letter};
  }
  \foreach \line in {36,...,70} {
     \begin{scope}[xshift=0cm,yshift=-(\line-36+1)*5mm]
       \foreach \letter/\position in {A/1,B/2,C/3,D/4,E/5} {
           \node at (-0.1,0) {\line};
           \node[fill=black!30,draw,circle,inner sep=3pt] at ({\position * 0.5},0) {};
           \node[fill=white,draw,circle,inner sep=2pt] at ({\position * 0.5},0) {};
       }
     \end{scope}
  }
\end{tikzpicture}\hspace{1 cm}
\begin{tikzpicture}[font=\tiny]
  \foreach \letter/\position in {A/1,B/2,C/3,D/4,E/5} {
    \node[inner sep=3pt] at ({\position * 0.5},0) {\letter};
  }
  \foreach \line in {71,...,105} {
     \begin{scope}[xshift=0cm,yshift=-(\line-71+1)*5mm]
       \foreach \letter/\position in {A/1,B/2,C/3,D/4,E/5} {
           \node at (-0.1,0) {\line};
           \node[fill=black!30,draw,circle,inner sep=3pt] at ({\position * 0.5},0) {};
           \node[fill=white,draw,circle,inner sep=2pt] at ({\position * 0.5},0) {};
       }
     \end{scope}
  }
\end{tikzpicture}\hspace{1 cm}
\begin{tikzpicture}[font=\tiny]
  \foreach \letter/\position in {A/1,B/2,C/3,D/4,E/5} {
    \node[inner sep=3pt] at ({\position * 0.5},0) {\letter};
  }
  \foreach \line in {106,...,140} {
     \begin{scope}[xshift=0cm,yshift=-(\line-106+1)*5mm]
       \foreach \letter/\position in {A/1,B/2,C/3,D/4,E/5} {
           \node at (-0.1,0) {\line};
           \node[fill=black!30,draw,circle,inner sep=3pt] at ({\position * 0.5},0) {};
           \node[fill=white,draw,circle,inner sep=2pt] at ({\position * 0.5},0) {};
       }
     \end{scope}
  }
\end{tikzpicture}\hspace{1 cm}


\end{center}
\vspace{1cm}\noindent\varhrulefill[0.4mm]\vspace{1mm}

\vspace{-3mm}\noindent\varhrulefill[0.4mm]\vspace{1mm}


\newpage\end{document}