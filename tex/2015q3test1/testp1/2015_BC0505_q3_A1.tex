
        \documentclass[10pt,brazil,a4paper]{exam}
        \usepackage[latin1]{inputenc}
        \usepackage[portuguese]{babel}
        \usepackage[dvips]{graphicx}
        %\usepackage{multicol}
        %\usepackage{shadow}
        %\usepackage{pifont}
        %\usepackage{listings}
        %\usepackage{fancyvrb}
        
        \newcommand*\varhrulefill[1][0.4pt]{\leavevmode\leaders\hrule height#1\hfill\kern0pt}
        
        \def\drawLines#1{{\color{cyan}\foreach \x in {1,...,#1}{\par\vspace{2mm}\noindent\hrulefill}}}
        
        \usepackage{enumitem}
        \usepackage{multirow}
        \usepackage{amsmath}
        \usepackage{changepage,ifthen}
        %\usepackage{boxedminipage}
        %\usepackage{theorem}
        \usepackage{verbatim}
        \usepackage{tabularx}
        %\usepackage{moreverb}
        \usepackage{times}
        %\usepackage{relsize}
        \usepackage{pst-barcode}
        \usepackage{tikz}
        \setlength{\textwidth}{185mm}
        \setlength{\oddsidemargin}{-0.5in}
        \setlength{\evensidemargin}{0in}
        \setlength{\columnsep}{8mm}
        \setlength{\topmargin}{-28mm}
        \setlength{\textheight}{265mm}
        \setlength{\itemsep}{0in}
        \begin{document}
        \pagestyle{empty}
        %\lstset{language=python}
        {\small
\makeatletter\renewcommand*\cleardoublepage{\ifodd\c@page \else\hbox{}\newpage\fi}
\makeatother
\cleardoublepage
\begin{table}[h]\centering
\begin{tabular}{|p{16mm}|p{16cm}|}
\hline\multirow{4}{*}{\hspace{-2mm}\includegraphics[width=2cm]{../../../figs/ufabc.eps}}
&\vspace{-2mm}\noindent\large\textbf{Universidade Federal do ABC}\\
&\noindent\textbf{Disciplina:} Processamento da Informa��o - BC0505\\
&\noindent\textbf{Professor(es):} Denise Goya, Itana Stiubiener, Francisco Zampirolli, Luiz Rozante, Monael Ribeiro\\
&\noindent\textbf{Quadrimestre:} 3/2015\hfill\textbf{Modalidade:} Semipresencial\hfill\textbf{Data:} 26/11/2015\hspace{-8mm}\\
\hline\end{tabular}
\end{table}
\vspace{-4mm}\small{
\noindent\textbf{Aluno:}  GABARITO - multipla-escolha
\hfill\textbf{Matr�cula:} 00000000
\hfill\textbf{Turma:} 2015$\_$BC0505$\_$q3$\_$A1
\hspace{-1mm}

\vspace{8mm}
\noindent\textbf{Ass:}\rule{11.5cm}{0.1pt}\hfill\hspace{1cm}
}

\vspace{4mm}\noindent Escreva 111 um programa para inverter os elementos pares que est�o nas posi��es �mpares de um vetor com X elementos, onde X � um inteiro definido pelo usu�rio.
 
 \drawLines{10} 


 \ \ \ 
 \newpage
\makeatletter\renewcommand*\cleardoublepage{\ifodd\c@page \else\hbox{}\newpage\fi}
\makeatother
\cleardoublepage
\begin{table}[h]\centering
\begin{tabular}{|p{16mm}|p{16cm}|}
\hline\multirow{4}{*}{\hspace{-2mm}\includegraphics[width=2cm]{../../../figs/ufabc.eps}}
&\vspace{-2mm}\noindent\large\textbf{Universidade Federal do ABC}\\
&\noindent\textbf{Disciplina:} Processamento da Informa��o - BC0505\\
&\noindent\textbf{Professor(es):} Denise Goya, Itana Stiubiener, Francisco Zampirolli, Luiz Rozante, Monael Ribeiro\\
&\noindent\textbf{Quadrimestre:} 3/2015\hfill\textbf{Modalidade:} Semipresencial\hfill\textbf{Data:} 26/11/2015\hspace{-8mm}\\
\hline\end{tabular}
\end{table}
\vspace{-4mm}\small{
\noindent\textbf{Aluno:}  GABARITO - multipla-escolha
\hfill\textbf{Matr�cula:} 00000000
\hfill\textbf{Turma:} 2015$\_$BC0505$\_$q3$\_$A1
\hspace{-1mm}

\vspace{8mm}
\noindent\textbf{Ass:}\rule{11.5cm}{0.1pt}\hfill\hspace{1cm}
}

\vspace{4mm}\noindent Escreva 333 um programa para inverter os elementos �mpares que est�o nas posi��es pares de um vetor com X elementos, onde X � um inteiro definido pelo usu�rio.
 
 \drawLines{10} 


 \ \ \ 
 \newpage
\makeatletter\renewcommand*\cleardoublepage{\ifodd\c@page \else\hbox{}\newpage\fi}
\makeatother
\cleardoublepage
\begin{table}[h]\centering
\begin{tabular}{|p{16mm}|p{16cm}|}
\hline\multirow{4}{*}{\hspace{-2mm}\includegraphics[width=2cm]{../../../figs/ufabc.eps}}
&\vspace{-2mm}\noindent\large\textbf{Universidade Federal do ABC}\\
&\noindent\textbf{Disciplina:} Processamento da Informa��o - BC0505\\
&\noindent\textbf{Professor(es):} Denise Goya, Itana Stiubiener, Francisco Zampirolli, Luiz Rozante, Monael Ribeiro\\
&\noindent\textbf{Quadrimestre:} 3/2015\hfill\textbf{Modalidade:} Semipresencial\hfill\textbf{Data:} 26/11/2015\hspace{-8mm}\\
\hline\end{tabular}
\end{table}
\vspace{-4mm}\small{
\noindent\textbf{Aluno:}  GABARITO - multipla-escolha
\hfill\textbf{Matr�cula:} 00000000
\hfill\textbf{Turma:} 2015$\_$BC0505$\_$q3$\_$A1
\hspace{-1mm}

\vspace{8mm}
\noindent\textbf{Ass:}\rule{11.5cm}{0.1pt}\hfill\hspace{1cm}
}

\vspace{4mm}\noindent O programa abaixo l� dois valores para as vari�veis X e Y, efetua a troca dos valores de forma que a vari�vel X passe a ter o valor de Y, e que a vari�vel Y passe a ter o valor de X. Complete a(s) instru��o(�es) "AQUI".
 \begin{verbatim}
 programa
 {
     funcao inicio()
     {
           real X, Y, aux
           leia (X, Y)
           AQUI
           escreva(X, Y)
      }
 }
 \end{verbatim}
 
 \drawLines{10} 


 \ \ \ 
 \newpage
}
\end{document}